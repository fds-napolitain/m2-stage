\documentclass[a4paper,12pt]{article}
\usepackage[utf8]{inputenc}
\usepackage[french]{babel}
\usepackage[T1]{fontenc}
\usepackage{graphicx}
\usepackage{hyperref}
\usepackage{float}
\usepackage{glossaries}
\usepackage{csquotes}

\usepackage[
backend=biber,
style=numeric,
sorting=ynt
]{biblatex}

\addbibresource{parties/biblio.bib}

\makeglossaries

\graphicspath{ {./image/} }

\usepackage{lmodern,tikz,lipsum}
\usepackage[a4paper,
            bindingoffset=0.2in,
            left=1in,
            right=1in,
            top=1in,
            bottom=1in,
            footskip=.25in]{geometry}

\newcommand\framethispage[1][2cm]{%
    \tikz[overlay,remember picture,line width=2pt]
    \draw([xshift=(#1),yshift=(-#1)]current page.north west)rectangle
         ([xshift=(-#1),yshift=(#1)]current page.south east);%
}

\usepackage{wrapfig}
\usepackage{threeparttable} 


\usepackage{lipsum}
    
\begin{document}

\newcommand\interline{0.3cm}
\thispagestyle{empty}
%\fbox{
%\begin{minipage}{\textwidth}
\begin{titlepage}
\framethispage[1.9cm]% le cadre est à 2 cm des bords de la feuille
\begin{center}

\vspace{1cm}
\begin{minipage}[b]{3.5cm}
\includegraphics[width=2cm]{LOGO_RVB_grand-UM.jpg}
\end{minipage}\begin{minipage}[b]{3.5cm}

\includegraphics[width=2cm]{lirmm.fr.png}
\end{minipage}\begin{minipage}[b]{3.5cm}

\includegraphics[width=2cm]{logo.png}
\end{minipage}\begin{minipage}[b]{3cm}

\includegraphics[width=2cm]{LOGO_DTP_INFORMATIQUE.jpg}
\end{minipage}
\vspace{1cm}

\Large{Université de Montpellier\\
\vspace{\interline}

Faculté des Sciences} \\
\vspace{2cm}

{\huge MASTER 2 INFORMATIQUE \\
\vspace{\interline}

Parcours IMAGINE \\}
\vspace{2cm}
{\huge \textbf{Automatisation du prétraitement de photos de mandrills}}\\
\vspace{2cm}
Effectué au LIRMM et au CNRS \\
\vspace{\interline}
du 31 janvier au 29 juillet 2022 \\
\vspace{\interline}
par Maxime BOUCHER\\
\vspace{2cm}
Tuteurs en entreprise: William Puech et Julien Renoult \\
\vspace{\interline}
Tuteur à l'université: Madalina Croitoru 
\end{center}
\end{titlepage}


\newglossaryentry{tensorflow}
{
    name=tensorflow,
    description={Tensorflow est une librairie Python de Google, écrite en C++, qui permet d'écrire des réseaux de neurones avec une API assez haut niveau}
}

\newglossaryentry{ReLU}
{
    name=ReLU,
    description={La fonction d'activation ReLU est une fonction composée, linéaire sur $[0, \infty]$ et nulle sur $[-\infty, 0]$. Elle est donc non linéaire et permet aux réseaux de neurones d'être plus précis}
}

\newglossaryentry{kanban}
{
    name=kanban,
    description={Un kanban est un tableau où figurent multiples colonnes à étiquettes. Le développeur saisit une tâche, par exemple, dans la colonne TODO et la place dans DOING. Les autres développeurs voient alors que cette dernière est déjà en cours d'accomplissement. Enfin, une fois la tâche finie, le développeur la place dans une colonne DONE. On peut étendre ce système de méthode AGILE avec un système de peer-reviewing, où une fois une tâche accomplie, elle doit être vérifiée par un pair avant d'être réellement acceptée}
}

\newglossaryentry{matrice de confusion}
{
    name=matrice de confusion,
    description={Une matrice de confusion permet d'étudier les prédictions d'un modèle. L'accuracy, seule, n'est pas assez utile pour juger un modèle de classification. Car un modèle peut, dans un contexte d'imbalance des classes, toujours prédire la même chose et avoir une excellente accuracy. La matrice de confusion affiche quant à elle l'efficacité du modèle par classe, et donc permet de savoir si le modèle est aussi efficace sur chacune d'entre elles ou non}
}

\newglossaryentry{transfer learning}
{
    name=transfer learning,
    description={Le transfer learning est une méthode en machine learning pour réutiliser le travail déjà effectué. En effet, un réseau de neurones apprend à déceler des features : par exemple, la forme des yeux. Une fois entrainé, on va sauvegarder ces poids et potentiellement le distribuer sur Internet pour que d'autres puissent extraire ces features en réutilisant les poids sauvegardés. Ensuite, on peut spécialiser le réseau de neurones pour l'adapter à un problème qui serait similaire}
}

\newglossaryentry{classification}
{
    name=Classification,
    description={Une classification est une résolution d'un problème en machine learning où l'on veut prédire si une donnée en entrée appartient à une classe ou une autre. C'est une sorte de régression avec une sortie discrète plutôt que continue. C'est notamment utilisée pour le tagging dans la photographie (cet animal est-il un chat ou un chien?)}
}

\newglossaryentry{docker}
{
    name=docker,
    description={Docker est un logiciel qui permet de lancer un processus isolé et restreint. Ainsi, on peut utiliser Docker pour lancer des logiciels avec leurs propres versions de librairies. Par ailleurs, Docker utilise généralement Linux en OS host, donc c'est adapté à une utilisation, que ce soit en développement, production, petits ou gros systèmes..}
}

\newglossaryentry{accuracy}
{
    name=accuracy,
    description={L'accuracy est une mesure qui décrit le taux de de prédictions correctes.
    \begin{equation}
        accuracy = \frac{nombre de prédictions correctes}{nombres d'échantillons}
    \end{equation}}
}

\newglossaryentry{XMP}
{
    name=XMP,
    description={Les métadonnées XMP sont des métadonnées liés intrinsèquement à Adobe Lightroom. Elle permet de rajouter des tags dans le champs Keywords, que l'on utilisera pour stocker tous les différents labels qui seront utilises pour nos modèles de machine learning}
}

\newglossaryentry{oversampling}
{
    name=oversampling,
    description={L'oversampling est une méthode qui implique la génération d'images pour égaliser le nombre d'échantillons entre différentes classes, dans le cadre d'une classification (ou même d'un régression en théorie même s'il n'y a pas de classes). Diverses méthodes existent, et j'en ai créée une qui semble être bien adapté à notre problème en particulier, qui est décrite dans les réalisations}
}

\newglossaryentry{courbe ROC}
{
    name=courbe ROC,
    description={La courbe ROC permet d'étudier ici le taux de vrais positifs en fonction du taux de faux positifs, pour chaque classe}
}

\newglossaryentry{SQLite}
{
    name=SQLite,
    description={Une base de données SQLite est une base de données relationnel sans serveur (basé sur un fichier seul à la manière d'un CSV par exemple). C'est assez facile à partager ou gérer dans un laboratoire non spécialisé en informatique, tout en ayant un avantage certain en performances et robustesse comparé à une solution comme des fichiers CSV}
}

\newglossaryentry{CSV}
{
    name=CSV,
    description={Un fichier comma separated values (CSV) est un tableau dont chaque colonne est séparer par une virgule, et chaque ligne par un retour à la ligne}
}

\newglossaryentry{dataset}
{
    name=dataset,
    description={Un dataset est un ensemble de données. Dans notre cas, un dataset est un ensemble d'images (de mandrills) et de leurs labels associés (FaceQual pour n'en citer qu'un). En machine learning, on doit avoir les valeurs X, ici les images, et les valeurs Y, ici leur label, pour ensuite que le modèle fasse une régression de la fonction et puisse ensuite, à partir d'une nouvelle image X et de son entraînement, prédire leur label Y}
}

\newpage

\newpage

\renewcommand{\contentsname}{Table des matières}

\tableofcontents

\newpage


\newpage

\section{Remerciements}
Je tiens à remercier Monsieur Julien Renoult, chargé de recherche au CNRS, qui a co-encadré mon stage, et m'a énormément guidé dans mes actions et m'a fait confiance tout au long de cette période.
\newline

Je remercie également M. William Puech, professeur des universités et chercheur au LIRMM, qui, en tant que co-encadrant du stage, a porté un regard critique et bienveillant sur mes travaux. Je n'oublie pas ses nombreux conseils.\newline

Je remercie particulièrement Mme Claudia Ximena Restrepo-Ortiz, ingénieure en bioinformatique, ma tutrice opérationnelle sur le projet, qui a dirigé mes expérimentations de manière optimale, et m'a aidé dans ma façon de présenter les résultats des expérimentations.\newline

Je remercie également Mme Sonia Mai Tieo, experte en Intelligence Artificielle au CNRS, pour ses nombreux conseils, et son aide dans l'appréhension de la discipline.\newline

J'adresse également mes remerciements à ma tutrice côté Université, Mme Madalina Croitoru, avec qui je faisais des points réguliers, qui m'ont permis d'aborder ce stage en toute confiance.\newline

Et enfin, je remercie mon référent pédagogique, M. Pierre Pompidor, pour son aide efficace dans les démarches de recherche de stage et son suivi. 
\newline

\newpage
\section{Introduction}
Il s'agit d'un stage dans le domaine de l'intelligence artificielle, co-encadré par deux laboratoires, le LIRMM et le CEFE, grâce à une bourse de Master interdisciplinaire du projet MUSE. \newline

Le projet MUSE « Montpellier Université d’Excellence » vise à faire émerger à Montpellier une université thématique de recherche intensive, internationalement reconnue pour son impact dans les domaines liés à l’agriculture, l’environnement et la santé.\newline

Le stage dure du 31 février au 29 juillet 2022, il se déroule au sein du projet "Mandrillus".\newline

Le projet Mandrillus, né en 2012 et dirigé par Mme Marie Charpentier (CNRS), se déroule principalement dans la réserve de Lékédi au Gabon, et a pour objectifs:
\begin{itemize}
    \item d'obtenir des informations sur la vie et l'écologie des Mandrills
    \item d'aider à la protection des Mandrills
    \item de soutenir les populations locales\newline
\end{itemize}


Depuis 2012, l’équipe a rassemblé plus de 30,000 portraits de mandrills issus d'une population de prêt de 300 individus. L'équipe sur le terrain produit aujourd'hui prêt de 2,000 nouvelles images par mois.\newline

L'objectif de ce stage est de concevoir des algorithmes qui permettront, au final, d'automatiser la reconnaissance des individus et l'annotation des nouvelles images. Pour cela, il faut d'abord concevoir les traitements intermédiaires qui permettront d'atteindre l'objectif final:  
\begin{itemize}
    \item évaluation de la qualité des photos, si le mandrill présent sur cette dernière est de face ou de profil, en vue d'un tri pour des usages a posteriori.
    \item évaluation de l'âge des mandrills à partir de leur photographie.
    \item éventuellement, mise au propre / uniformisation du code pour une meilleur maintenance
\end{itemize}
\newpage
\section{Présentation de l'entreprise} 

\subsection{Le CNRS - CEFE}

Le centre d'écologie fonctionnelle et évolutive (CEFE) est un centre de recherche du CNRS dédié à l'écologie. Pour citer leur résumé sur leur site web \cite{cefe}: 
Le projet du CEFE vise à comprendre la dynamique, le fonctionnement et l’évolution du vivant, de «la bactérie à l’éléphant », et « du génome à la planète ». Il s’appuie sur trois ambitions : [1] comprendre le monde vivant pour anticiper ce que sera demain, [2] conduire à des innovations et répondre aux attentes de la société ; [3] pratiquer une science « rassembleuse » et diverse dans ses approches disciplinaires. Les questions de recherche sont posées dans un contexte marqué par la prégnance des changements planétaires, le développement de nouvelles technologies de manipulation du vivant, et l’exigence croissante de la société pour la recherche.\newline
(...)\newline
Le CEFE est organisé en quatre départements scientifiques entourés de plates-formes techniques communes. 

\begin{itemize}
    \item Ecologie Evolutive et Comportementale
    \item Dynamique et Conservation de la Biodiversité
    \item Ecologie Fonctionnelle
    \item Interactions, Ecologie et Sociétés \\
\end{itemize}

Au sein du CEFE, l'aile E3CO (Ecologie Evolutive Empirique, Communication \& Coopération) est quant à elle consacrée à l'étude de l'écologie évolutive et comportementale, dédié aux êtres vivants.
L'équipe dirigée par Julien Renoult travaille, entre autres, sur le projet Mandrillus, auquel j'ai participé.

\subsection{Le LIRMM}

Le laboratoire d'informatique, de robotique et microélectronique de Montpellier qui est une collaboration entre le CNRS et l'université de Montpellier.
J'ai été plus précisément affecté à l'équipe ICAR, qui travaille sur la compression d'images, la sécurité et la 3D principalement.\newline

L’équipe ICAR (Image \& Interaction) regroupe des chercheurs des deux départements Robotique et Informatique autour de la thématique « image » et plus généralement des données visuelles. Elle est composée actuellement de neufs permanents, universitaires et CNRS mais compte aussi dans ses collaborateurs réguliers, plusieurs médecins hospitalo-universitaires du CHU utilisant l’imagerie médicale, des chercheurs en télédétection du laboratoire TETIS ou en modélisation pour l’agronomie du CIRAD.\newline

L’équipe ICAR développe des thèmes de recherche associant l’interaction et le traitement des données visuelles telles que les images 2D, 3D, multi-spectrales (nD), les vidéos ou les séquences d’images nD+t et les objets 3D que ce soit sous forme de maillages 3D ou de modélisations paramétriques.\newline

L’équipe est structurée suivant 4 axes de recherche : 
\begin{itemize}
    \item Analyse \& traitement
    \item Sécurité Multimédia
    \item Modélisation \& Visualisation
    \item Intelligence Artificielle pour les données visuelles
\end{itemize}


\section{Présentation de la mission (facultatif, nécessaire si la présentation du sujet dans l'introduction ne suffit pas) }


\section{Environnement technique} 
Je travaille sur un ordinateur fourni par le CNRS, avec, lorsque je suis au lirmm, une connexion à distance par Remote Desktop Protocol (RDP). J'utilise également mon ordinateur fixe sur lequel je me connecte en SSH. \\

À mon arrivé, le contrôle de versions n'existait pas. On a donc mis en place un projet Github versionné par Git, le standard pour le contrôle de versions décentralisés. \\

Pour le deep learning, Tensorflow (créé par Google) 2.X sera utilisé. C'est une des librairies majeurs en machine learning, avec PyTorch (créé par Facebook) pour concurrent. \\

Le choix d'ajouter Docker dans une partie du travail permet de s'affranchir en partie du problème de l'environnement technique, c'est-à-dire que l'environnement de l'utilisateur est différent de celui de développement et de production (dans notre cas, la production ne représente pas réellement quelque chose de séparé du développement mais on peut facilement imaginer qu'on veuille utiliser un ordinateur plus puissant, voire un supercalculateur, pour entrainer le réseau de neurones. Dans ce cas là, Docker permettrait que le code fonctionne sans problèmes).

\section{Veille technologique ou appropriation scientifique (facultatif)} 
Durant la période entre les derniers partiels et le début du stage, j'ai travaillé les différents tutoriels Tensorflow, et j'ai commencé à effleurer la théorie concernant le machine learning (descente de gradient...)\newline

Puis, pendant le stage, j'ai étudié des méthodes pour résoudre des problématiques spécifiques, comme le bruit de labellisation, ou encore l'ordonnancement des classes, en répertoriant des méthodes et en  mettant en pratique quelques unes. \newline

J'ai réalisé un état de l'art de la prédiction de l'âge à partir d'images de face (d'humains).

\section{Travaux effectués} 

\subsection{\Gls{classification} des images par qualité, face profils}
L'objectif fixé à la première visio conférence a été déterminé : faire de la \gls{classification} sur la qualité des photographies des mandrills.\\

J'ai donc commencé par faire du \gls{transfer learning} avec MobileNet V2, un réseau neuronal convolutif (CNN)  de Google adapté aux mobiles (c'est-à-dire, très rapide et plutôt performant) pour essayer d'adapter un modèle avec les poids ImageNet pour la \gls{classification} de la qualité des photographies. Notre \gls{dataset} ressemble aux images ci-dessous, avec FaceQual0-3 en label de qualité, 3 étant la meilleure qualité.

\begin{figure}
    \centering
    \includegraphics[width=300pt]{imgs/qualité/cr1/dataset.png}
    \caption{Exemples d'images avec leur label qualité correspondant}
    \label{fig:my_label}
\end{figure}

Les premiers résultats semblent bons à première vue : 85\% d'\gls{accuracy} sur une \gls{classification} simple.

\begin{figure}
    \centering
    \includegraphics[width=300pt]{imgs/qualité/cr1/resultat1.png}
    \caption{Résultat}
    \label{fig:my_label}
\end{figure}

\begin{figure}
    \centering
    \includegraphics[width=300pt]{imgs/qualité/cr1/prediction1.png}
    \caption{Prédictions}
    \label{fig:my_label}
\end{figure}

Mais les résultats sont trompeurs, en effet, le jeu de données est très déséquilibré, avec 10000 images de qualité FaceQual2 et 6000 de qualité FaceQual3 contre 250 et 1000 de qualité FaceQual0 et FaceQual1. Le modèle pourrait donc prédire tout le temps FaceQual2 et avoir 57\% d'\gls{accuracy} par défaut.\\

Il faut donc gérer ce déséquilibre, et deux approches nous paraissent intéressantes :\\
\begin{itemize}
    \item dégrader des images de bonne qualité (downscale puis upscale et léger floutage) pour génerer des photos de mauvaises qualité (une mauvaise photo souvent a un manque de détails et/ou est flou).
    \item pondérer les classes de jeu de données pendant l'entrainement (accorder plus d'importance donc là où on aurait moins d'échantillons).
\end{itemize}

Voici un exemple de ce que pourrait être une dégradation d'images par rapport aux premières images :

\begin{figure}
    \centering
    \includegraphics[width=300pt]{imgs/qualité/cr1/augmentation1.png}
    \caption[Images dégradées]{Voici les mêmes images que sur la figure 1, mais avec une opération de dégradation par floutage et downscale/upscale. On peut voir qu'une photo de bonne qualité normalement ne l'est plus.}
    \label{fig:my_label}
\end{figure}

\begin{figure}
    \centering
    \includegraphics[width=300pt]{imgs/qualité/cr9/generate.png}
    \caption[Code pour dégrader les images]{Fonctions permettant de générer les images de qualité mauvaises (0) à partir des images de très bonne qualités (3). Le but étant de réduire l'impact de l'imbalance des classes, on créé autant d'images de qualité 0 qu'il n'en faut pour égaliser la classe 0 et la classe 1.}
    \label{fig:my_label}
\end{figure}

Évidemment, on peut faire varier le niveau de floutage et de dégradation de détails. On pourrait également imaginer appliquer un fort taux de compression JPEG ou augmenter le bruit artificiellement pour obtenir des images de basse qualité réalistes (tout cela étant des facteurs de basse qualité).\\

Ensuite, j'ai travaillé sur la \gls{matrice de confusion} et le \gls{graphe ROC} et commencer à comparer de manière plus complète les modèles.\\

Nous démarrons avec un \gls{transfer learning} par VGG16/ImageNet suivi de deux couches complètement connectés, de type Dense (chaque neurone, de chaque couche, est inter connecté) et avec une activation \gls{ReLU} (non linéaire), et par enfin une couche de type Dense avec une activation SoftMax (probabilités, entre 0 et 1).

Nous obtenons sur 10 epochs un rappel à 0.8 pour les 2 classes les plus représentés (FaceQual2/3) et 0.65 pour les autres (FaceQual0/1). Cela veut donc dire qu'une proportion correcte initiale d'échantillons est correctement classé.

\begin{figure}
    \centering
    \includegraphics[width=300pt]{imgs/qualité/cr2/vgg16_000_confusion.png}
    \caption[Matrice de confusion pour une 1re classification]{\Gls{Matrice de confusion} représentant le taux d'échantillons prédis par classe pour chacuns des labels connus. Par exemple, 64\% des échantillons FaceQual0 ont correctement été prédis, et 31\% ont été prédis avec une classe d'écart (FaceQual1).}
    \label{fig:my_label}
\end{figure}

\begin{figure}
    \centering
    \includegraphics[width=300pt]{imgs/qualité/cr2/vgg16_000_roc.png}
    \caption[Courbe ROC associé pour une 1re classification]{La \gls{courbe ROC} montre que les classes avec un plus faible taux de vrais positifs (les images de mauvaises qualités) ont également un taux plus faible de faux positifs. Cela veut dire qu'il est rare qu'une image de bonne qualité soit considéré comme de mauvaise qualité, mais qu'il arrive qu'une image de mauvaise qualité soit considéré de bonne qualité.}
    \label{fig:my_label}
\end{figure}

Pour améliorer la situation, problématique sûrement à cause du déséquilibre des données, nous pouvons essayer d'abord la pondération des classes. Cela permet en effet d'améliorer largement le résultat de la classe la plus sous représentée (FaceQual0), avec un poids d'importance de 17. Seulement, une partie de l'apprentissage est troublée, d'une part car la classe FaceQual1 est partiellement confondue par la classe FaceQual0 et également du fait que beaucoup d'échantillons FaceQual2 sont classés en FaceQual0 (5\% des FaceQual2, représentant une bonne partie sachant que FaceQual2 possède énormément plus d'images).


\begin{center}
    \includegraphics[width=210pt]{imgs/qualité/cr2/vgg16_010_confusion.png}
    \includegraphics[width=210pt]{imgs/qualité/cr2/vgg16_010_roc.png}
\end{center}

Une piste est alors d'opérer un \gls{oversampling} a priori sur la classe FaceQual0 pour avoir des poids de classes plus adéquats (moins grands). \\

Une fois testé, nous trouvons un \gls{rappel} supérieur à la situation initiale sans cette fois-ci occasionner de dommages collatéraux. Le poids de la classe FaceQual0 après l'\gls{oversampling} par copie/dégradation des images FaceQual3 est autour de 4, soit 4 fois moins qu'avant \gls{oversampling}.

\begin{center}
    \includegraphics[width=210pt]{imgs/qualité/cr2/vgg16_011_confusion.png}
    \includegraphics[width=210pt]{imgs/qualité/cr2/vgg16_011_roc.png}
\end{center}

\subsection{Prédictions de classes ordonnées}
Pour résoudre ce problème de \gls{classification}, il est tentant de réduire le problème à une régression, puisque les classes sont corrélés : les classes 1FaceQual0-3 correspondent à une suite d'images de qualité croissante.
Enfin, pour les classifier de manière discrète, il suffirait d'arrondir les prédictions alors continus.
\begin{center}
    \includegraphics[width=300pt]{imgs/qualité/cr6/confusion-regression.png}
\end{center}

Le résultat n'est cependant pas aussi bon. Cela est sûrement en partie dû au fait que les valeurs ne sont pas prédites dans l'intervalle [0; 3] mais $[-\infty; +\infty]$.

On peut supposer affiner les résultats en améliorant les méthodes liés à la normalisation, ou alors en retournant au modèle de \gls{classification} d'avant avec cross entropy.

Une autre méthode à explorer pour résoudre un problème de \gls{classification} multi classes ordonné, est d'utiliser un encodage one-hot particulier tel que décrit sur cette \href{https://stats.stackexchange.com/questions/140061/how-to-set-up-neural-network-to-output-ordinal-data}{réponse} et  \href{https://arxiv.org/pdf/1901.07884.pdf}{papier}.
J'ai donc créé un générateur personnalisé d'images, qui prend des \gls{batch} d'images avec leur label et retourne normalement les mêmes \gls{batch} avec les mêmes labels encodés sous la nouvelle forme :\\
\begin{gather*}
0 \longrightarrow [0, 0, 0]\ (au\ lieu\ de\ [1, 0, 0, 0])\\
1 \longrightarrow [1, 0, 0]\ (au\ lieu\ de\ [0, 1, 0, 0])\\
2 \longrightarrow [1, 1, 0]\ (au\ lieu\ de\ [0, 0, 1, 0])\\
3 \longrightarrow [1, 1, 1]\ (au\ lieu\ de\ [0, 0, 0, 1])\\
\end{gather*}
Par ailleurs, au lieu d'utiliser une dernière couche à k neurones et pour activation softmax, j'utilise k-1 neurones avec pour activation sigmoid. Enfin, pour avoir le numéro de la classe prédite, on fait la somme des probabilités, ce qui donne par exemple : $[1,1,0] \longrightarrow 2
[img onehot]$

\subsection{Base de données des métadonnées}
Avant le commencement du stage, les métadonnées étaient répartis sur plusieurs fichiers \gls{CSV}. Cela induisait donc une certaine dureté pour y appliquer des modifications, ainsi qu'une rigidité pour utiliser les fichiers.\\

Alors nous nous sommes poser la question s'il pourrait être intéressant de créer une base de données pour organiser mieux le lien entre les métadonnées d'une image originale, d'une image portrait, d'un individu mandrill, etc...
Pour moi, le choix le plus cohérent était de partir sur une base de données \gls{SQLite}, dite "file-based" (basé seulement sur le fichier). Ainsi, cela reste relativement simple à utiliser pour un centre de recherche non spécialisé en informatique puisqu'il n'y a pas de serveur, où tout le monde n'aurait pas envie de gérer ce dernier. Ce n'est pas non plus nécessaire car il n'y a pas réellement d'accès concurrent à cette base de données : elle sert avant tout pour les modèles d'entraînement ensuite.

\begin{center}
    \includegraphics[width=450pt]{imgs/qualité/cr8/sqlite_schema.png}
\end{center}

\subsection{Utilisation de \Gls{docker}}
Pour améliorer la portabilité de tous les scripts, j'ai mis en place Docker ainsi que Docker Compose. Docker est un logiciel basé sur la virtualisation (notion de conteneurs, isolés du système hôte), qui permet de lancer un script avec son environnement (système d'exploitation et dépendances) "figé", de sorte qu'il fonctionne de manière reproductible dans le temps et partout.\\

En effet, lorsqu'on doit installer un environnent de travail \gls{tensorflow}, il y a besoin de respecter une matrice de compatibilité des dépendances, ce qui peut s'avérer complexe avec les évolutions des différentes librairies dans le temps. De plus, l'interfaçage avec les drivers de cartes graphiques (GPU) ajoute une couche de complexité dans cette mise en place. 

Dans la logique \Gls{docker}, on peut écrire des Dockerfile afin de construire des images étendues personnalisées en partant d'image de base existantes (qui peuvent elles même être des images étendues).

On peut également utiliser Docker Compose pour orchestrer les lancements des conteneurs basé sur les images. Ceci est utile pour stocker les configurations plutôt que d'utiliser les nombreux paramètres en lignes de commande.

Voici des exemples de mes Dockerfile et docker-compose.yml : \\

Dans le Dockerfile, nous précisions l'image de base (un OS ou, ici, un OS déjà étendu avec une version spécifique de Python). Ensuite nous construisons l'image étendue, en y copiant les scripts locaux et en installant les dépendances enregistrées dans un fichier requirements.txt (qui est le standard pour le gestionnaire de dépendances python pip). Enfin la commande ENTRYPOINT indique le point de départ de l'application, et CMD les paramètres par défaut, qui sont pour leur part surchargeables.
\begin{center}
    \includegraphics[width=450pt]{imgs/qualité/cr9/dockerfile.png}
\end{center}

Dans docker-compose.yml, nous pouvons orchestrer le lancement de plusieurs conteneurs Docker issus de Dockerfile pour qu'ils soient dépendants les uns des autres (les scripts d'IA nécessitants la base de donnée par exemple) ou simplement les lancer individuellement. Cela permet également de décrire pour les services, les volumes (bind mounts) directement dans un fichier de configuration, plutôt que comme paramètres de la commande Docker. Sans ces volumes montés, les données produites par les scripts ne seraient pas persistantes, car les conteneurs lancés avec Docker sont éphémères.
\begin{center}
    \includegraphics[width=200pt]{imgs/qualité/cr9/compose.png}
\end{center}

Voici un exemple de build des images, et de run de l'image de la base de donnée (le service db). Comme le docker-compose indique l'utilisation de bind mounts, le répertoire /app/Documents/Metadata/metadata.sqlite dans le container n'est autre que le fichier [projet]/Documents/Metadata/metadata.sqlite sur le système hôte (persistant).
\begin{center}
    \includegraphics[width=450pt]{imgs/qualité/cr9/build.png}
    \includegraphics[width=450pt]{imgs/qualité/cr9/run.png}
\end{center}

\subsection{Modifications des keywords}
Les keywords sont un champs de métadonnées IPTC/\gls{XMP}, c'est-à-dire de métadonnées de photographies (jpeg), lues notamment par Adobe Lightroom. A mon arrivée, le champs Keywords contenait une liste d'éléments comme 1FaceQual3 pour indiquer une photo de bonne qualité, 1FaceQual2 pour une photo de qualité un peu moindre, ainsi que 1FaceView pour indiquer si le mandrill est de face, etc. \\ 

Seulement ce n'était pas très robuste à la lecture, si un des champs (comme la qualité) venait à manquer, puisque la liste était ensuite dans le mauvais ordre. On a donc changé vers début mai l'inscription de la manière suivante :
\begin{itemize}
    \item 1FaceQual0 devient FaceQual:0
    \item 1FaceQual1 devient FaceQual:1
    \item 1FaceQual2 devient FaceQual:2
    \item 1FaceQual3 devient FaceQual:3
    \item Le reste devient FaceQual:-1
\end{itemize}
De même pour tous les autres mots clés (FaceView pour n'en citer qu'un).

\subsection{Gridsearch}
La méthode de recherche gridsearch consiste à énumérer toutes les possibilités pour un certain nombre d'hyperparamètres en machine learning. Cela veut dire, lancer l'entraînement avec par exemple, un batchsize de 32, puis de 64, puis répéter le processus en changeant également le learning rate.
\begin{center}
    \includegraphics[width=450pt]{imgs/qualité/cr10/gridsearch.png}
    \includegraphics[width=450pt]{imgs/qualité/cr10/gridsearch2.png}
\end{center}
Comme on le voit ci-dessus : les commandes pour lancer les entraînements sont générés (et le script les lancent séquentiellement) en énumérant tous les hyperparamètres que l'on a spécifiés.

\section{Conclusion}
Le stage m'a permis d'intégrer la méthode scientifique dans mon travail : rigueur au niveau des comparaisons et résultats. C'est donc une leçon qui a une très longue portée car cela me sera utile toute ma vie et peu importe le domaine. \\

Par ailleurs, j'en ai appris un peu plus sur le machine learning et les statistiques qui, là encore, présentent un intérêt dans plusieurs disciplines scientifiques. \\

J'ai également utilisé mes connaissances en gestion de projet, en particulier agile, en conservant un kanban personnel, et en faisant des réunions assez régulières (mais pas de mêlées quotidiennes), ainsi qu'en gérant le projet github.


\section{Glossaire}
\clearpage
\printglossary[title={Glossaire}]


\listoffigures

\section{Bibliographie} 
\printbibliography[
heading=subbibintoc,
title={ }
] 

\end{document}
